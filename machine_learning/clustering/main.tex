\documentclass{article}

\usepackage[letterpaper,margin=1in]{geometry}
\usepackage{enumitem}
\setlist{noitemsep,topsep=0.03in} % reduce spacing in lists
\usepackage{microtype}

\usepackage{amsfonts}
\usepackage{amsmath}
\usepackage{amssymb}

\newenvironment{annotation}{\begin{quotation}\noindent\textit{Annotation:} }{\end{quotation}}

\usepackage{natbib}
\usepackage{bibentry}

\title{\textbf{Annotated Bibliography \\ on Clustering}}
\author{Chelsea H. Komlo}
\date{Winter 2019}

\begin{document}

\bibliographystyle{abbrvnat}

\maketitle

\section*{Annotated References}

\begin{itemize}

\item
	\bibentry{TODOCITE
	\begin{annotation}
    Discusses statistical aspects of clustering. Introduces the idea that the
    more sample points that are available, the more reliable the clustering
    should be.

    Good indicators of whether a statistical data set has been clustered
    appropriately include convergence proofs and stability considerations.
    Specifically, 1) the clustering should converge as the more data points are
    added, which indicates stability. However, it is important to rule out the
    trivial result and ensure that clustering algorithms select interesting
    results .Thus, 2) the algorithm should also demonstrate flexibility, which
    requires sensitivity to sample variations.

    Stability requires that a good clustering algorithm are stable with respect
    to the clustering process, and would not change much by drawing another
    sample.
	\end{annotation}

\item
	\bibentry{TODOCITE2
	\begin{annotation}
    Addresses clustering quality measurements as the object to be axiomatized,
    as opposed to the theory of clustering itself, which was proven to be
    impossible by Kleinberg. Demonstrates that principles such as those
    proposed by Kleinberg can apply to clustering quality measurements without
    resulting in inconsistency.

    By focusing on evaluating the quality of a given data clustering, it is
    possible to identify concrete principles and anxioms to evaluate clustering
    algorithms themselves.

    The aim of clustering is to unconver meaningful groups of data. After a
    clustering has been performed, a user must look at this clustering to
    determine if it is "good" or not. Thus, clustering quality of measurement
    (CQMS) can help to quantify how good is any specific clustering.

    Kleinberg's anxioms- 1) function scale invariance, requiring that the
    output of a clustering algorithm is invariant to uniform scaling of the
    input, 2) function consistency, requiring that if within-cluster distances
    are decreased and between cluster distances are increased, then the output
    of the clustering algorithm should not change, and 3) function richness,
    requiring that modifying the distance function results in any partition of
    the underlying data set.

    These axioms, when reformulated as CQMS, prove useful to evaluate
    clustering algorithms and do not lead to inconsistency.


	\end{annotation}
\end{itemize}

\end{document}

\documentclass{article}

\usepackage[letterpaper,margin=1in]{geometry}
\usepackage{enumitem}
\setlist{noitemsep,topsep=0.03in} % reduce spacing in lists
\usepackage{microtype}

\usepackage{amsfonts}
\usepackage{amsmath}
\usepackage{amssymb}

\newenvironment{annotation}{\begin{quotation}\noindent\textit{Annotation:} }{\end{quotation}}

\usepackage{natbib}
\usepackage{bibentry}

\title{\textbf{Annotated Bibliography \\ on Clustering}}
\author{Chelsea H. Komlo}
\date{Winter 2019}

\begin{document}

\bibliographystyle{abbrvnat}

\maketitle

\section*{Annotated References}

\begin{itemize}

\item
	\bibentry{TODOCITE
	\begin{annotation}
    Discusses statistical aspects of clustering. Introduces the idea that the
    more sample points that are available, the more reliable the clustering
    should be.

    Good indicators of whether a statistical data set has been clustered
    appropriately include convergence proofs and stability considerations.
    Specifically, 1) the clustering should converge as the more data points are
    added, which indicates stability. However, it is important to rule out the
    trivial result and ensure that clustering algorithms select interesting
    results .Thus, 2) the algorithm should also demonstrate flexibility, which
    requires sensitivity to sample variations.

    Stability requires that a good clustering algorithm are stable with respect
    to the clustering process, and would not change much by drawing another
    sample.
	\end{annotation}
\end{itemize}

\end{document}

\documentclass{article}

\usepackage[letterpaper,margin=1in]{geometry}
\usepackage{enumitem}
\setlist{noitemsep,topsep=0.03in} % reduce spacing in lists
\usepackage{microtype}

\usepackage{amsfonts}
\usepackage{amsmath}
\usepackage{amssymb}

\newenvironment{annotation}{\begin{quotation}\noindent\textit{Annotation:} }{\end{quotation}}

\usepackage{natbib}
\usepackage{bibentry}

\title{\textbf{Annotated Bibliography \\ on Private Information Retrieval}}
\author{Chelsea H. Komlo}
\date{Spring 2019}

\begin{document}

\bibliographystyle{abbrvnat}

\maketitle

\section*{Categorization}

\section*{Annotated References}

\begin{itemize}

\item
  \bibentry{Gasarch:2004{
	\begin{annotation}
    Private information retrieval is when a user wants to query a database for
    information but not reveal to the server what information they want to
    seek. The simplest way to achieve this is for the server to send them
    \textbf{all} of the information. The question in PIR is how to achieve this
    goal more efficiently.

    One way to achieve this is by viewing the database as a $\sqrt{n}$ x
    $\sqrt{n}$ database and using properties of xor. By generating
    carefully-crafited bit strings and sending both to two separate databases,
    who xor the entire contents of their database, the user can xor the
    responses, revealing only the desired index, without reavealing this to the
    databases (Theorem 3.2).
	\end{annotation}
\item
  \bibentry{Blackburn:2016{
	\begin{annotation}
    Presents a PIR scheme that minimizes the download complexity while also
    minimizing the amount of storage required for servers.

    Examples of real-world PIR systems include Popcorn and Splinter, achieving
    latencies less than 1.2 seconds for realistic work loads.

    Proving optimal download complexity- when the number of records is greater
    than 2, the download complexity must be as long as $R+1$, where $R$ is the
    length of the record. For an $n$-server PIR query, the lowerboad is
    $\frac{n}{n-1}R$ when $k$ (the number of records) is sufficiently large.
	\end{annotation}
\end{itemize}

\end{document}

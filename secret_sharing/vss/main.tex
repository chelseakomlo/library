\documentclass{article}

\usepackage[letterpaper,margin=1in]{geometry}
\usepackage{enumitem}
\setlist{noitemsep,topsep=0.03in} % reduce spacing in lists
\usepackage{microtype}

\usepackage{amsfonts}
\usepackage{amsmath}
\usepackage{amssymb}

\newenvironment{annotation}{\begin{quotation}\noindent\textit{Annotation:} }{\end{quotation}}

\usepackage{natbib}
\usepackage{bibentry}

\title{\textbf{Annotated Bibliography \\ on Verifiable Secret Sharing}}
\author{Chelsea H. Komlo}
\date{Winter 2019}

\begin{document}

\bibliographystyle{abbrvnat}

\maketitle

\section*{Categorization}

\section*{Annotated References}

\begin{itemize}

\item
	\bibentry{SAC:NN2005
	\begin{annotation}
    Adds the definition of a mobile adversary, where a mobile adversary can
    move between players over time, controlling an ever-changing subset of the
    players. Proactive schemes are supposed to mitigate mobile players as
    refreshing shares mitigates the threat that the adversary will control $4$
    players at any point of time in an epoch. This requires that the system be
    set up in such a way that the adversary not have enough time to control $t$
    or more shares in an epoch.

    Several well-known VSS schemes are broken when considered in the setting of
    a mobile adversary. For example, Feldman's VSS scheme with a proactive
    addition is broken (this holds true for Pederson's scheme as well). Other
    known Unconditionally Secure VSS schemes also fail with a mobile adversary.

    Identifies conditions for security of Proactive VSS, namely that for
    protocols to be secure against b cheating players, polynomials of $k-1$
    should be used, and recommend that $b \leq k$. For computationally
    secure (b, k, n) proactive VSS is secure if and only if the number of
    cheating players + the threshold number is less than the total number of
    participants- $b + k \leq n$.

    Thus, when updates are made via a commitment to zero, only $t-2$ shares
    can be lost per epoch in reality, as if $t-1$ shares are lost in one epoch,
    an adversary could work backwards from the next epoch and determine the
    secret.
	\end{annotation}
\end{itemize}

\end{document}
